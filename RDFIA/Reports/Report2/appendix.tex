\chapter{Appendix}
\section{Transfer Learning}
\graphicspath{{figs/2a/}}
\subsection{Activation maps} \label{appendix:activation_maps}

\begin{figure}[H]
    \centering
    \includegraphics*[width=0.95\textwidth]{activation_dog}
    \caption{Activation maps obtanied after the first convolutional layer, on the Shih Tzu dog image.}
    \label{fig:activation_dog}
\end{figure} 

\begin{figure}[H]
    \centering
    \includegraphics*[width=0.95\textwidth]{activation_komondor}
    \caption{Activation maps obtanied after the first convolutional layer, on the Komondor dog image.}
    \label{fig:activation_komondor}
\end{figure} 

\begin{figure}[H]
    \centering
    \includegraphics*[width=0.95\textwidth]{activation_komondor_or_mop}
    \caption{Activation maps obtanied after the first convolutional layer, on the "Komondor or mop?" image.}
    \label{fig:activation_komondor_or_mop}
\end{figure} 

\begin{figure}[H]
    \centering
    \includegraphics*[width=0.95\textwidth]{activation_wombat}
    \caption{Activation maps obtanied after the first convolutional layer, on the wombat image.}
    \label{fig:activation_wombat}
\end{figure} 

\begin{figure}[H]
    \centering
    \includegraphics*[width=0.95\textwidth]{activation_dog_meme}
    \caption{Activation maps obtanied after the first convolutional layer, on the dog (meme) image.}
    \label{fig:activation_dog_meme}
\end{figure} 

\subsection{Going further: VisionTransformer} \label{appendix:vit}

\begin{figure}[H]
    \centering
    \begin{subfigure}{0.95\textwidth}
        \includegraphics[width=\textwidth]{original_images_vit}
        \caption{}
        \label{subfig:original_images_vit}
    \end{subfigure}
    \begin{subfigure}{0.95\textwidth}
        \includegraphics[width=\textwidth]{transformed_images_vit}
        \caption{}
        \label{subfig:transformed_images_vit}
    \end{subfigure}
    \begin{subfigure}{0.95\textwidth}
        \includegraphics[width=\textwidth]{prediction_plots_vit}
        \caption{}
        \label{subfig:prediction_plots_vit}
    \end{subfigure}
    \caption{Performance evaluation of VisionTransformer pre-trained on ImageNet (IMAGENET1K\_V1 weights): (a) Original images, (b) Transformed images normalized and resized to $224 \times 224$, and (c) Top 5 predicted classes.}
    \label{fig:vit_scratch}
\end{figure}

\begin{figure}[H]
    \centering
    \begin{subfigure}{0.95\textwidth}
        \includegraphics[width=\textwidth]{original_images_vit_bis}
        \caption{}
        \label{subfig:original_images_vit_bis}
    \end{subfigure}
    \begin{subfigure}{0.95\textwidth}
        \includegraphics[width=\textwidth]{transformed_images_vit_bis}
        \caption{}
        \label{subfig:transformed_images_vit_bis}
    \end{subfigure}
    \begin{subfigure}{0.95\textwidth}
        \includegraphics[width=\textwidth]{prediction_plots_vit_bis}
        \caption{}
        \label{subfig:prediction_plots_vit_bis}
    \end{subfigure}
    \caption{Performance evaluation of VisionTransformer pre-trained on ImageNet (IMAGENET1K\_SWAG\_E2E\_V1 weights): (a) Original images, (b) Transformed images normalized and resized to $224 \times 224$, and (c) Top 5 predicted classes.}
    \label{fig:vit_refined}
\end{figure}

\section{Generative Adversarial Networks}
\graphicspath{{figs/2de/}}

\subsection{Longer epochs} \label{appendix:longer_epochs}

\begin{figure}[H]
    \centering

    \begin{subfigure}{0.45\textwidth}
        \centering
        \includegraphics[width=0.95\linewidth]{longer_epochs/lossD.png}
        \caption{}
        \label{subfig:longer_epochs/lossD}
    \end{subfigure}%
    \begin{subfigure}{0.45\textwidth}
        \centering
        \includegraphics[width=0.95\linewidth]{longer_epochs/lossG.png}
        \caption{}
        \label{subfig:longer_epochs/lossG}
    \end{subfigure}

    \begin{subfigure}{0.45\textwidth}
        \centering
        \includegraphics[width=0.95\linewidth]{longer_epochs/D_G_z.png}
        \caption{}
        \label{subfig:longer_epochs/D_G_z}
    \end{subfigure}%
    \begin{subfigure}{0.45\textwidth}
        \centering
        \includegraphics[width=0.95\linewidth]{longer_epochs/D_x.png}
        \caption{}
        \label{subfig:longer_epochs/D_x}
    \end{subfigure}

    \begin{subfigure}{0.45\textwidth}
        \centering
        \includegraphics[width=0.95\linewidth]{longer_epochs/epochs.png}
        \caption{}
        \label{subfig:longer_epochs/epochs}
    \end{subfigure}%

    \caption{Training curves: (a) $loss(D)$, (b) $loss(G)$, (c) $D(G(z))$, (d) $D(x)$ and (e) the number of epochs by the number of iterations.}
    \label{fig:longer_epochs_losses}
\end{figure}

\subsection{Influence of weight initialization.} \label{appendix:init}

\begin{figure}[H]
    \centering

    \begin{subfigure}{0.45\textwidth}
        \centering
        \includegraphics[width=0.95\linewidth]{init/lossD.png}
        \caption{}
        \label{subfig:init/lossD}
    \end{subfigure}%
    \begin{subfigure}{0.45\textwidth}
        \centering
        \includegraphics[width=0.95\linewidth]{init/lossG.png}
        \caption{}
        \label{subfig:init/lossG}
    \end{subfigure}

    \begin{subfigure}{0.45\textwidth}
        \centering
        \includegraphics[width=0.95\linewidth]{init/D_G_z.png}
        \caption{}
        \label{subfig:init/D_G_z}
    \end{subfigure}%
    \begin{subfigure}{0.45\textwidth}
        \centering
        \includegraphics[width=0.95\linewidth]{init/D_x.png}
        \caption{}
        \label{subfig:init/D_x}
    \end{subfigure}

    \begin{subfigure}{0.45\textwidth}
        \centering
        \includegraphics[width=0.95\linewidth]{init/epochs.png}
        \caption{}
        \label{subfig:init/epochs}
    \end{subfigure}%

    \caption{Training curves: (a) $loss(D)$, (b) $loss(G)$, (c) $D(G(z))$, (d) $D(x)$ and (e) the number of epochs by the number of iterations.}
    \label{fig:init_losses}
\end{figure}

\subsection{Influence of learning rate} \label{appendix:gan_lr}

\begin{figure}[H]
    \centering

    \begin{subfigure}{0.45\textwidth}
        \centering
        \includegraphics[width=0.95\linewidth]{lr/lossD.png}
        \caption{}
        \label{subfig:lr/lossD}
    \end{subfigure}%
    \begin{subfigure}{0.45\textwidth}
        \centering
        \includegraphics[width=0.95\linewidth]{lr/lossG.png}
        \caption{}
        \label{subfig:lr/lossG}
    \end{subfigure}

    \begin{subfigure}{0.45\textwidth}
        \centering
        \includegraphics[width=0.95\linewidth]{lr/D_G_z.png}
        \caption{}
        \label{subfig:lr/D_G_z}
    \end{subfigure}%
    \begin{subfigure}{0.45\textwidth}
        \centering
        \includegraphics[width=0.95\linewidth]{lr/D_x.png}
        \caption{}
        \label{subfig:lr/D_x}
    \end{subfigure}

    \begin{subfigure}{0.45\textwidth}
        \centering
        \includegraphics[width=0.95\linewidth]{lr/epochs.png}
        \caption{}
        \label{subfig:lr/epochs}
    \end{subfigure}%

    \caption{Training curves: (a) $loss(D)$, (b) $loss(G)$, (c) $D(G(z))$, (d) $D(x)$ and (e) the number of epochs by the number of iterations.}
    \label{fig:lr_losses}
\end{figure}

\subsection{Influence of $ngf$ and $ndf$} \label{appendix:ngf_ndf}

\begin{figure}[H]
    \centering

    \begin{subfigure}{0.45\textwidth}
        \centering
        \includegraphics[width=0.95\linewidth]{ngf/8/lossD.png}
        \caption{}
        \label{subfig:ngf/8/lossD}
    \end{subfigure}%
    \begin{subfigure}{0.45\textwidth}
        \centering
        \includegraphics[width=0.95\linewidth]{ngf/8/lossG.png}
        \caption{}
        \label{subfig:ngf/8/lossG}
    \end{subfigure}

    \begin{subfigure}{0.45\textwidth}
        \centering
        \includegraphics[width=0.95\linewidth]{ngf/8/D_G_z.png}
        \caption{}
        \label{subfig:ngf/8/D_G_z}
    \end{subfigure}%
    \begin{subfigure}{0.45\textwidth}
        \centering
        \includegraphics[width=0.95\linewidth]{ngf/8/D_x.png}
        \caption{}
        \label{subfig:ngf/8/D_x}
    \end{subfigure}

    \begin{subfigure}{0.45\textwidth}
        \centering
        \includegraphics[width=0.95\linewidth]{ngf/8/epochs.png}
        \caption{}
        \label{subfig:ngf/8/epochs}
    \end{subfigure}%

    \caption{Training curves for $ngf=8$: (a) $loss(D)$, (b) $loss(G)$, (c) $D(G(z))$, (d) $D(x)$ and (e) the number of epochs by the number of iterations.}
    \label{fig:ngf/8_losses}
\end{figure}

\begin{figure}[H]
    \centering

    \begin{subfigure}{0.45\textwidth}
        \centering
        \includegraphics[width=0.95\linewidth]{ngf/128/lossD.png}
        \caption{}
        \label{subfig:ngf/128/lossD}
    \end{subfigure}%
    \begin{subfigure}{0.45\textwidth}
        \centering
        \includegraphics[width=0.95\linewidth]{ngf/128/lossG.png}
        \caption{}
        \label{subfig:ngf/128/lossG}
    \end{subfigure}

    \begin{subfigure}{0.45\textwidth}
        \centering
        \includegraphics[width=0.95\linewidth]{ngf/128/D_G_z.png}
        \caption{}
        \label{subfig:ngf/128/D_G_z}
    \end{subfigure}%
    \begin{subfigure}{0.45\textwidth}
        \centering
        \includegraphics[width=0.95\linewidth]{ngf/128/D_x.png}
        \caption{}
        \label{subfig:ngf/128/D_x}
    \end{subfigure}

    \begin{subfigure}{0.45\textwidth}
        \centering
        \includegraphics[width=0.95\linewidth]{ngf/128/epochs.png}
        \caption{}
        \label{subfig:ngf/128/epochs}
    \end{subfigure}%

    \caption{Training curves for $ngf=128$: (a) $loss(D)$, (b) $loss(G)$, (c) $D(G(z))$, (d) $D(x)$ and (e) the number of epochs by the number of iterations.}
    \label{fig:ngf/128_losses}
\end{figure}

\begin{figure}[H]
    \centering

    \begin{subfigure}{0.45\textwidth}
        \centering
        \includegraphics[width=0.95\linewidth]{ndf/8/lossD.png}
        \caption{}
        \label{subfig:ndf/8/lossD}
    \end{subfigure}%
    \begin{subfigure}{0.45\textwidth}
        \centering
        \includegraphics[width=0.95\linewidth]{ndf/8/lossG.png}
        \caption{}
        \label{subfig:ndf/8/lossG}
    \end{subfigure}

    \begin{subfigure}{0.45\textwidth}
        \centering
        \includegraphics[width=0.95\linewidth]{ndf/8/D_G_z.png}
        \caption{}
        \label{subfig:ndf/8/D_G_z}
    \end{subfigure}%
    \begin{subfigure}{0.45\textwidth}
        \centering
        \includegraphics[width=0.95\linewidth]{ndf/8/D_x.png}
        \caption{}
        \label{subfig:ndf/8/D_x}
    \end{subfigure}

    \begin{subfigure}{0.45\textwidth}
        \centering
        \includegraphics[width=0.95\linewidth]{ndf/8/epochs.png}
        \caption{}
        \label{subfig:ndf/8/epochs}
    \end{subfigure}%

    \caption{Training curves for $ndf=8$: (a) $loss(D)$, (b) $loss(G)$, (c) $D(G(z))$, (d) $D(x)$ and (e) the number of epochs by the number of iterations.}
    \label{fig:ndf/8_losses}
\end{figure}


\begin{figure}[H]
    \centering

    \begin{subfigure}{0.45\textwidth}
        \centering
        \includegraphics[width=0.95\linewidth]{ndf/128/lossD.png}
        \caption{}
        \label{subfig:ndf/128/lossD}
    \end{subfigure}%
    \begin{subfigure}{0.45\textwidth}
        \centering
        \includegraphics[width=0.95\linewidth]{ndf/128/lossG.png}
        \caption{}
        \label{subfig:ndf/128/lossG}
    \end{subfigure}

    \begin{subfigure}{0.45\textwidth}
        \centering
        \includegraphics[width=0.95\linewidth]{ndf/128/D_G_z.png}
        \caption{}
        \label{subfig:ndf/128/D_G_z}
    \end{subfigure}%
    \begin{subfigure}{0.45\textwidth}
        \centering
        \includegraphics[width=0.95\linewidth]{ndf/128/D_x.png}
        \caption{}
        \label{subfig:ndf/128/D_x}
    \end{subfigure}

    \begin{subfigure}{0.45\textwidth}
        \centering
        \includegraphics[width=0.95\linewidth]{ndf/128/epochs.png}
        \caption{}
        \label{subfig:ndf/128/epochs}
    \end{subfigure}%

    \caption{Training curves for $ndf=128$: (a) $loss(D)$, (b) $loss(G)$, (c) $D(G(z))$, (d) $D(x)$ and (e) the number of epochs by the number of iterations.}
    \label{fig:ndf/128_losses}
\end{figure}

\subsection{Trying CIFAR-10 dataset} \label{appendix:cifar10_32}

\begin{figure}[H]
    \centering

    \begin{subfigure}{0.45\textwidth}
        \centering
        \includegraphics[width=0.95\linewidth]{cifar10/32/lossD.png}
        \caption{}
        \label{subfig:cifar10/32/lossD}
    \end{subfigure}%
    \begin{subfigure}{0.45\textwidth}
        \centering
        \includegraphics[width=0.95\linewidth]{cifar10/32/lossG.png}
        \caption{}
        \label{subfig:cifar10/32/lossG}
    \end{subfigure}

    \begin{subfigure}{0.45\textwidth}
        \centering
        \includegraphics[width=0.95\linewidth]{cifar10/32/D_G_z.png}
        \caption{}
        \label{subfig:cifar10/32/D_G_z}
    \end{subfigure}%
    \begin{subfigure}{0.45\textwidth}
        \centering
        \includegraphics[width=0.95\linewidth]{cifar10/32/D_x.png}
        \caption{}
        \label{subfig:cifar10/32/D_x}
    \end{subfigure}

    \begin{subfigure}{0.45\textwidth}
        \centering
        \includegraphics[width=0.95\linewidth]{cifar10/32/epochs.png}
        \caption{}
        \label{subfig:cifar10/32/epochs}
    \end{subfigure}%

    \caption{Training curves: (a) $loss(D)$, (b) $loss(G)$, (c) $D(G(z))$, (d) $D(x)$ and (e) the number of epochs by the number of iterations.}
    \label{fig:cifar10_32_curves}
\end{figure}

\subsection{Generating higher resolution images on CIFAR-10} \label{appendix:cifar10_64}

\begin{figure}[H]
    \centering

    \begin{subfigure}{0.45\textwidth}
        \centering
        \includegraphics[width=0.95\linewidth]{cifar10/64_nz100/lossD.png}
        \caption{}
        \label{subfig:cifar10/64_nz100/lossD}
    \end{subfigure}%
    \begin{subfigure}{0.45\textwidth}
        \centering
        \includegraphics[width=0.95\linewidth]{cifar10/64_nz100/lossG.png}
        \caption{}
        \label{subfig:cifar10/64_nz100/lossG}
    \end{subfigure}

    \begin{subfigure}{0.45\textwidth}
        \centering
        \includegraphics[width=0.95\linewidth]{cifar10/64_nz100/D_G_z.png}
        \caption{}
        \label{subfig:cifar10/64_nz100/D_G_z}
    \end{subfigure}%
    \begin{subfigure}{0.45\textwidth}
        \centering
        \includegraphics[width=0.95\linewidth]{cifar10/64_nz100/D_x.png}
        \caption{}
        \label{subfig:cifar10/64_nz100/D_x}
    \end{subfigure}

    \begin{subfigure}{0.45\textwidth}
        \centering
        \includegraphics[width=0.95\linewidth]{cifar10/64_nz100/epochs.png}
        \caption{}
        \label{subfig:cifar10/64_nz100/epochs}
    \end{subfigure}%

    \caption{Training curves: (a) $loss(D)$, (b) $loss(G)$, (c) $D(G(z))$, (d) $D(x)$ and (e) the number of epochs by the number of iterations.}
    \label{fig:cifar10_64_nz100_curves}
\end{figure}

\subsection{Conditional GAN} \label{appendix:cDCGAN}

\begin{figure}[H]
    \centering

    \begin{subfigure}{0.45\textwidth}
        \centering
        \includegraphics[width=0.95\linewidth]{cDCGAN/lossD.png}
        \caption{}
        \label{subfig:cDCGAN/lossD}
    \end{subfigure}%
    \begin{subfigure}{0.45\textwidth}
        \centering
        \includegraphics[width=0.95\linewidth]{cDCGAN/lossG.png}
        \caption{}
        \label{subfig:cDCGAN/lossG}
    \end{subfigure}

    \begin{subfigure}{0.45\textwidth}
        \centering
        \includegraphics[width=0.95\linewidth]{cDCGAN/D_G_z.png}
        \caption{}
        \label{subfig:cDCGAN/D_G_z}
    \end{subfigure}%
    \begin{subfigure}{0.45\textwidth}
        \centering
        \includegraphics[width=0.95\linewidth]{cDCGAN/D_x.png}
        \caption{}
        \label{subfig:cDCGAN/D_x}
    \end{subfigure}

    \begin{subfigure}{0.45\textwidth}
        \centering
        \includegraphics[width=0.95\linewidth]{cDCGAN/epochs.png}
        \caption{}
        \label{subfig:cDCGAN/epochs}
    \end{subfigure}%

    \caption{Training curves: (a) $loss(D)$, (b) $loss(G)$, (c) $D(G(z))$, (d) $D(x)$ and (e) the number of epochs by the number of iterations.}
    \label{fig:cDCGAN_curves}
\end{figure}
