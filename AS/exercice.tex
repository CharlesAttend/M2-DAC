\documentclass{article}
\usepackage[utf8]{inputenc}
\usepackage[a4paper, margin=2.5cm]{geometry}
\usepackage{graphicx}
% \usepackage[french]{babel}

\usepackage[default,scale=0.95]{opensans}
\usepackage[T1]{fontenc}
\usepackage{amssymb} %math
\usepackage{amsmath}
\usepackage{amsthm}
\usepackage{systeme}
\usepackage{bbm}

\usepackage{hyperref}
\hypersetup{
    colorlinks=true,
    linkcolor=blue,
    filecolor=magenta,      
    urlcolor=cyan,
    pdftitle={Overleaf Example},
    % pdfpagemode=FullScreen,
    }
\urlstyle{same} %\href{url}{Text}

\theoremstyle{plain}% default
\newtheorem{thm}{Théorème}[section]
\newtheorem{lem}[thm]{Lemme}
\newtheorem{prop}[thm]{Proposition}
\newtheorem*{cor}{Corollaire}
%\newtheorem*{KL}{Klein’s Lemma}

\theoremstyle{definition}
\newtheorem{defn}{Définition}[section]
\newtheorem{exmp}{Exemple}[section]
% \newtheorem{xca}[exmp]{Exercise}

\theoremstyle{remark}
\newtheorem*{rem}{Remarque}
\newtheorem*{note}{Note}
%\newtheorem{case}{Case}



\title{Apprentissage Statistique}
\author{Exercice du diapo}
\date{S1-2023}

\begin{document}
\maketitle

\section{Supervised classification}
\subsection{Excercice}
Exercice 1 : 
\begin{align*}
    L^* &= 1 - P(g^*(X) = Y) \\
        &= 1 - \mathbb{E} [ g^*(X) = 1] P( Y = 1 | X) \\
        &= 1 - \mathbb{E} [ \mathbbm{1}_{P(Y=1 | X)} + \mathbbm{1}_{g^*(X) = 0} P(Y=0 | X)] \\
        &= 1 - \mathbb{E} [ \mathbbm{1}_{\eta (X) > 1/2} \eta (X) + \mathbbm{1}_{\eta (X) \leq  1/2} (1 - \eta (X))]
\end{align*}
    
Exercice 2
\begin{align*}
    L^ * &= 1 - P(g^*(X) = Y) \\
        &= 1 - \mathbb{E} [P(g^*(X) = Y | X)] \\
        &= 1 - \mathbb{E} [ \max ( 1 - \eta (X) , \eta (X))] \\
        &= 1 + \mathbb{E} [ \min (\eta (X) - 1 , - \eta (X))] \\
        &= \mathbb{E} [ \min  (\eta (X) , 1 - \eta (X))]
\end{align*}

Exercice 3:
Si $ L*(X) = 0 $ ça veut dire que c'est un processus déterministe. Que $ Y $  a un lien déterministe avec $ X $ 
\[
    P(g^*(X) \neq  Y) = 0 \Rightarrow  Y = g^*(X) as. 
.\]

\[
    Y = \phi (X) \Rightarrow P(Y \neq \phi (X) ) = 0 \Rightarrow L* = 0
.\]

\subsection{Statistical Learning}
Exercice diapo 18 : $ consistency \Leftrightarrow L(g_n) \to ^{L^1} L* \Leftrightarrow L(g_n) \to ^{ \mathbb{P}} L* $ 

\begin{itemize}
    \item Pour la convergence L1 (je crois)
    \begin{align*}
        &P(g_n(X) \neq Y | \mathcal{D}_n) \to ^{L1} L* \\
        &\mathbb{E}[ P(g_n (X) \neq  Y | \mathcal{D} _n) - L*] \\
        &= \mathbb{E} [ P(g_n(X) \neq Y [ \mathcal{D}_n ) - L*])] \\
        &= P(g_n(X) \neq Y ) - L* \\
        & \to 0 (?)
    \end{align*}
    \item Pour la convergence en proba dans le sens non instinctif, on veut montrer que
    \begin{align*}
        &Z_n \to ^{\mathbb{P}} 0 \\
        &\left| Z_n \right|  \leq 1 \text{ car proba} \\
        &\text{ alors } Z_n \to ^{L1} 0
    \end{align*}
    Preuve : \begin{align*}
        &\mathbb{E} [ \left| Z_n \right| ] = \mathbb{E} [ Z_n  | \mathbbm{1}_{\left| Z_n \right| > \epsilon  }] + \mathbb{E} [ Z_n | \mathbbm{1}_{\left| Z_n \right| \leq \epsilon }] \\
        &\leq P(\left| Z_n \right| > \epsilon )
    \end{align*}
\end{itemize}


\subsection{Exercice TD}
\subsubsection{Exercice 1}
$ (X,Y) \in \mathbb{R}\times [0,1], X \sim \mathcal{U}([-2, 2]), ect $ 
\begin{align*}
    Y &= \mathbbm{1}_{X< 0, U \leq 2} + \mathbbm{1}_{X > 0, U >1} \\
    \eta &= P(Y=1 | X) \\
        &= P(X < 0, u \leq 2 | X) + P(X > 0, U > 1 | X) \\
        &= \mathbbm{1}_{X < 0} P(U \leq 2) + \mathbbm{1}_{X > 0} P(U > 1) \\
        &= \mathbbm{1}_{X < 0} \frac{2}{10} + \mathbbm{1}_{X > 0} \frac{9}{10} \\
        g*(X) = \mathbbm{1}_{\eta (X) >\frac{1}{2}} = \mathbbm{1}_{X > 0}
\end{align*}
Bayes error 
\begin{align*}
    L* &= P(g*(X) \neq Y) \\
        &= \frac{1}{2} - \frac{1}{2} \mathbb{E}[\left| 2 \eta (X) - 1 \right| ] \\
        &= \frac{1}{2} - \frac{1}{2} \mathbb{E}[\left| \frac{4}{10} \mathbbm{1}_{X < 0} + \frac{18}{10} \mathbbm{1}_{X > 0} - 1 \right| ] \\
        &= \frac{1}{2} - \frac{1}{2} \mathbb{E}[\left| \frac{6}{10} \mathbbm{1}_{X < 0} + \frac{8}{10} \mathbbm{1}_{X > 0} \right| ] \\
        &= \frac{1}{2} - \frac{1}{2}* \frac{6}{10} * \frac{1}{2} - \frac{1}{2}* \frac{8}{10} * \frac{1}{2} \\
        &= \frac{3}{20}
\end{align*}

\subsubsection{Exercice 2}
\begin{enumerate}
    \item \begin{align*}
        L* &= \mathbb{E}[\min(\eta (X) , 1 - \eta (X))] \\
            &= \mathbb{E}[\min (\frac{x}{c+x} , \frac{c}{c+x})] \\ 
            &= \mathbb{E}[ \frac{\min (x,c)}{ c+x}]
    \end{align*}
    \item $ f_x(x) = \frac{\mathbbm{1}_{x \in [0, \alpha c]}}{\alpha c} $ 
    \begin{align*}
        L* &= \int_{0}^{\alpha c} \frac{\min (x,c)}{x + c} \frac{dx}{\alpha c} \\
            &= \int_{0}^{c} \frac{x}{x+c} \frac{dx}{\alpha c} \\ 
            &= \frac{1}{\alpha } \log_{} ( \frac{(\alpha +1) e}{4})
    \end{align*} 
    \item exercice, étudier la fonction ou \begin{itemize}
        \item $ f $ continue sur $ [1, +\infty ] $ 
        \item $ \lim_{\alpha  \to \infty} f(\alpha ) = 0 $ 
    \end{itemize}
    Docn $ f $ admet un maximum 
\end{enumerate}

\subsubsection{Exerice 3}
$ X \sim (T,B,E) \sim \mathcal{E}(1) $ \\
densité d'une loi exp $ Z \sim \mathcal{E}(1) $ \begin{align*}
    f_Z(z) &= e^{ -z \mathbbm{1}_{\mathbb{R}_+} (z)} \\
    F_Z(t) &= (1 - e^{-t}) \mathbbm{1}_{\mathbb{R}_+}(t) \\
    G_z(t) &= \mathbbm{1}_{\mathbb{R}_-}+e^{-t} \mathbbm{1}_{\mathbb{R}_+}(t)
\end{align*}
\begin{enumerate}
    \item $ Y $ est une fonction de $ X $ donc $ L* = 0 $. C'est déterministe
    \item \begin{align*}
        P(Y=1 | T, B) &= P(T + B + E < 7 | T, B) \\
            &= F_Z (7 - T - B) \text{ car } E \bot (T, B)
            &= (1 - e^{- ( 7 - T - B)}) \mathbbm{1}_{7 - T - B > 0}
    \end{align*} 
    \item \begin{align*}
        &g^*(T,B) = \mathbbm{1}_{\eta (T, B) > 1/2} \\
        & \eta (T, B) > 1/2 \\
        \Leftrightarrow& 1 - e^{T + B - 7} > \frac{1}{2} \\
        \Leftrightarrow& - \ln 2 > T + B - 7 \\
        \Leftrightarrow& T + B < 7 - \ln 2
    \end{align*}
    Donc $ g*(T,B) = \mathbbm{1}_{T + B < 7 - \ln 2} $ 
    \item $ T, B \sim \mathcal{E}(1) $  et $ T \bot B $ donc $ T + B \sim \gamma (2,1) $, $ f_{T + B} (u) = ue^{-u} \mathbbm{1}_{\mathbb{R}_+ (u)} $ 
    \item \begin{align*}
        L^* = P(g^*(T,B) \neq  Y) &= P(g^*(T, B) = 1, Y = 0) + P(g^* (T,B) = 1, Y = 0) \\
            &= P(T + B < 7 - \ln 2, T + B+ E \geq 7) + P(T+B \geq 7 - \ln 2, T+B+E < 7)\\
            &= a + b \\
        a &= \mathbb{E} [ P( T+B < 7 - \ln 2 , T+B+3 \geq 7 | T,B)]\\
        &= \mathbb{E}[ \mathbbm{1}_{T+B < 7 - \ln 2} G(7 - T - B)] \\
        &= \mathbb{E} [ \mathbbm{1}_{T +B < 7 - \ln 2} e^{T + B - 7}] \\ 
        &= \int_{0}^{7 - \ln 2}e^{u - 7 } u e^{-u} du \\
        &= e^{-7} [ \frac{u^2}{2}]^{7 - \ln 2}_{0} \\
        &= e^{-7} \frac{(7 - \ln 2)^2}{2} \\
        b &= \text{ same }\\
        a + b &= e^{-7} ( \frac{(7 - \ln 2)^2}{2} + 2 ( 8 - \ln 2) - 8 - \frac{7^2}{2} + \frac{(7 - \ln 2)^2}{2})
    \end{align*}
    \item \begin{align*}
        P(Y = 0) &= P(T+B+E \geq 7) \\
            &= P(\gamma (3) \geq 7) \\
            &= \int_{7}^{+\infty } \frac{1}{2} u^2 e^{-u}du \\
            &= 0.029
    \end{align*}
\end{enumerate}



\end{document}