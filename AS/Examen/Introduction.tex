\chapter{Introduction}
Avant l'introduction de ResNet en 2015 par He et al., l'architecture GoogLeNet était le dernier gagnant des challenges de vision par ordinateur. Cette architecture avait été développée pour pallier les problèmes d'apprentissage liés à l'augmentation de la profondeur de VGG, une autre architecture proéminente.

Un réseau plus profond peut offrir de meilleures performances dans certaines conditions, mais il est aussi sujet à des problèmes tels que l'explosion ou l'évanouissement du gradient de la loss. Durant la rétropropagation, les grandes ou petites valeurs de gradient peuvent s'amplifier à travers les couches du réseau, entraînant un gradient bien plus grand ou plus petit dans les dernières couches par rapport aux premières. Cet effet est multiplicatif et dépend donc de la profondeur du réseau.

Pour un réseau d'une profondeur $L$, on modélise ces états cachés de dimension $d$ par une séquence $(h_k)_{1 \leq k \leq L}$ avec $h_k \in \mathbb{R}^d, \forall 0 \leq k \leq L$. L'explosion du gradient peut être décrite mathématiquement par, avec une forte probabilité, $\left\| \frac{\partial \mathcal{L}}{\partial h_0} \right\| \gg \left\| \frac{\partial \mathcal{L}}{\partial h_L} \right\|$, où $\mathcal{L}$ représente la loss et $\left\| \cdot \right\|$ la norme euclidienne.

GoogLeNet, bien qu'offrant des une légère amélioration des performance par rapport à VGG, était encore relativement complexe et sa profondeur comparable à celle de VGG, passant de 22 à 16 couches. En 2015, ResNet a introduit un modèle allant jusqu'à 152 couches, divisant par deux le nombre d'erreurs de GoogLeNet. Son innovation réside dans l'intégration de \textit{skip connections} entre les couches successives, facilitant le passage du gradient au sein du réseau. Mathématiquement, cela donne la relation récurrente suivante pour la séquence $(h_k)_{1 \leq k \leq L}$ :
\[
    h_{k+1} = h_k + f(h_k, \theta_{k+1})
.\]

où $f(\cdot, \theta_{k+1})$ représente les transformations effectuées par la couche $k$ et paramétrées par $\theta_{k+1} \in \mathbb{R}^p$.

Les ResNets sont devenus la base de nombreux modèles d'apprentissage profond de pointe, s'étendant au-delà du traitement d'images pour inclure des domaines tels que le traitement du langage naturel et l'apprentissage par renforcement. L'idée des \textit{skip connections} a inspiré de nombreuses autres architectures et est devenue une pratique courante dans la conception des réseaux neuronaux profonds.

\begin{figure}[htbp]
    \centering
    % \includegraphics[width=.9\textwidth]{figs/}
    \caption{Illustration du modèle ResNet avec la présence de \textit{skip connections} dans chaque bloc.}
    \label{fig:resnet}
\end{figure}

Malgré ces avancées, ResNet rencontre toujours des problèmes de gradient durant l'apprentissage. La méthode traditionnelle pour contrer cela est la normalisation des états cachés après chaque couche (\textit{batch normalization}). Cependant, cette approche a un coût computationnel et dépend fortement de la taille du \textit{batch}. Une alternative est d'incorporer un facteur d'échelle $\alpha_L$ devant le terme résiduel, conduisant au modèle suivant :
\begin{equation}\label{resnet_equation}
    h_{k+1} = h_k + \alpha_L f(h_k, \theta_{k+1})
\end{equation}
Le choix de $\alpha_L$ est crucial et dépend naturellement de la profondeur $L$ du réseau. Il assure que la variance du signal reste stable lors de sa propagation à travers les couches. Cependant, il n'existe actuellement ni preuve formelle ni justification mathématique solide pour le choix de ce facteur de régularisation.

Dans ce cours, nous examinerons les fondements mathématiques pour choisir la valeur de $\alpha_L$ en fonction de $L$ et de la distribution initiale des poids, dans le but d'éviter les problèmes d'apprentissage. Deux axes principaux d'étude seront abordés :
\begin{enumerate}
    \item Le facteur $\alpha_L$ à l'initialisation : L'initialisation des paramètres est cruciale pour la phase d'apprentissage d'un modèle et influe même sur ses capacités de généralisation. Une mauvaise initialisation peut entraîner une divergence ou une disparition rapide du gradient, voire un blocage dans l'apprentissage. L'étude du rôle de $\alpha_L$ lors de l'initialisation est donc pertinente. Nous considérerons que, à l'initialisation, les poids de chaque couche $(\theta_k)_{1 \leq k \leq L}$ sont choisis de manière indépendante et identique selon une loi, typiquement gaussienne ou uniforme sur $\mathbb{R}^p$. 
    \item L'approche continue : Bien que le réseau neuronal soit constitué de nombreuses couches distinctes, l'ensemble du réseau peut être considéré comme une fonction, lorsqu'il y a suffisament de neurones. L'idée centrale des équations différentielles neuronales est de traiter les couches distincts du réseau comme continues, en supposant que chaque couche subit de petits changements. Ainsi, l'entrée de la couche suivante est considérée comme le résultat de l'intégrale de l'entrée de la couche précédente. Cela peut être comparé au mouvement d'un projet, où le déplacement dans la deuxième seconde peut être approximé par la déplacement dans la première seconde plus la vitesse dans la première seconde (multipliée par l'intervalle de temps d'une seconde).
    Les équations différentielles neuronales ont ouvert une nouvelle direction pour la théorie et la pratique de l’apprentissage profond et ont été appliquées à la classification d’images, aux séries chronologiques et à d’autres domaines. Tous les réseaux d'apprentissage profond avec des connexions résiduelles peuvent être exprimés approximativement par des équations différentielles neuronales
\end{enumerate}