% arguelles v2.3.0
% author: Michele Piazzai
% contact: michele.piazzai@uc3m.es
% license: MIT

\documentclass[compress,12pt]{beamer}
\usepackage{hyperref}
\usepackage{tikz}
\usepackage{nicefrac}

\newcommand{\xmark}{%
\tikz[scale=0.15] {
    \draw[line width=0.7,line cap=round] (0,0) to [bend left=6] (1,1);
    \draw[line width=0.7,line cap=round] (0.2,0.95) to [bend right=3] (0.8,0.05);
}}
\newcommand{\cmark}{%
\tikz[scale=0.15] {
    \draw[line width=0.7,line cap=round] (0.25,0) to [bend left=10] (1,1);
    \draw[line width=0.8,line cap=round] (0,0.35) to [bend right=1] (0.23,0);
}}
% \usepackage{emoji}
% \setemojifont{TwemojiMozilla}

\usetheme{Arguelles}
\title{Explainable AI is Dead, Long Live Explainable AI!}
\subtitle{Hypothesis-driven Decision Support using Evaluative AI}
\event{}
\date{}
\author{Charles Vin}
\institute{Sorbonne Université - 21216136}
% \email{charles.vin@outlook.fr}
% \homepage{charles.vin}
% \github{username}

\begin{document}

\frame[plain]{\titlepage}

\Section{Introduction}
\begin{frame}
      \frametitle{Quick summary}
      Paradigm Shift in AI Decision Support \\
      $\rightarrow$ Evaluative AI Concept \\
      Goals:
      \begin{itemize}
            \item Human-centered Approach
            \item Going Beyond Recommendations
            \item \textbf{Mitigating Over-Reliance}
            \item Support for Hypothesis Evaluation
            \item Machine-in-the-Loop Paradigm
      \end{itemize}
\end{frame}

\subsection{Over/Under reliance}
\begin{frame}
      \frametitle{Over/Under-reliance}
      \framesubtitle{Definitions}
      \begin{itemize}
            \item \textbf{Over-reliance}: Decision makers accept a machine recommendations, even when it is wrong, but would be rejected if coming from a human. \begin{itemize}
                  \item \textit{The machine "must be right" because it's a machine}
            \end{itemize}
            \item \textbf{Under-reliance}: Machine outputs are consistently rejected, even when it is correct, but would be accepted if coming from a human. 
      \end{itemize}
      $ \Rightarrow  $ Problems after deployement: \\ 
      AI systems ignored OR over-reliance related problems.
\end{frame}

\begin{frame}
      \frametitle{Over/Under-reliance}
      \framesubtitle{Causes}
      \begin{itemize}
            \item Over-reliance: Automation bias ;
            \item Under-reliance: Algorithmic aversion.
      \end{itemize}
      When adding XAI tools for more explaination \\
      $ \Rightarrow $ Confirmation bias (called fixation in the paper).
\end{frame}

\begin{frame}
      \frametitle{Over/Under-reliance}
      \framesubtitle{Solutions}
      \begin{itemize}
            \item Cognitive forcing \begin{itemize}
                  \item Eg. forcing people to give a decision before seeing a recommendation ;
                  \item Slightly mitigated overreliance, but not enought to lead to a statistically significant differences ;
                  \item Least prefered method by participant : people not wanting to exert mental energy.
            \end{itemize}
            \item Changing the XAI framework % {\fontspec{Symbola}\symbol{"1F914}}
      \end{itemize}
\end{frame}

\begin{frame}
      \frametitle{What makes a good decisions?}
      % \framesubtitle{Subtitle here}
      In a simple way: \begin{itemize}
            \item Identify options
            \item Compare options
            \item Choose an option
      \end{itemize}
      In a less simple way: the 10 "cardinal decision issue" outlined by Yates and Potworowski \begin{itemize}
            \item Needs, mode, Investment, Options, Possibilities, Judgements, Value, Trade-offs, Acceptability, Implementation
      \end{itemize}
\end{frame}

\begin{frame}
      \frametitle{What makes a good decisions support system?}
      \framesubtitle{Summed up}
      \begin{itemize}
            \item Options: Help to identify options, well as help to narrow down the list of feasible or realistic options
            \item Possibilities: Help to to identify possible outcomes 
            \item Judgement \& Value: Help to judge which outcomes are most likely and what will be the positive and negative impacts
            \item Trade-offs: Help to make trade-offs on the above criteria for each options
            \item Understandable: Help to understand how and why the tools works as it does, and when it fails
      \end{itemize}
\end{frame}


\section{Current XAI}

\begin{frame}[standout]
      \centering\large
      Does current decision support align with those criteria?
\end{frame}

\begin{frame}
      \frametitle{Giving recommendations with no explanatory information}
      \begin{columns}[T] % align columns
            \begin{column}{.75\textwidth}
                  \begin{figure}[htbp]
                        \centering
                        Figure 3
                  \end{figure}
                  A model of giving recommendations for decision support. 
                  
                  This assumes that decision makers will carefully consider recommendations. However, empirical evidence suggests this is not the case.
            \end{column}%
            \hfill%
            \begin{column}{.25\textwidth}
                  \scriptsize
                  \begin{itemize}
                        \item[\xmark] Options
                        \item[$ \nicefrac{1}{n} $ ] Possibilities
                        \item[$ \nicefrac{1}{n} $ ] Judgement \& Value
                        \item[\xmark] Trade-offs
                        \item[\xmark] Understandable
                  \end{itemize}
            \end{column}%
      \end{columns}
\end{frame}

\begin{frame}
      \frametitle{Giving recommendations with explanatory information}
      \begin{columns}[T] % align columns
            \begin{column}{.75\textwidth}
                  \begin{figure}[htbp]
                        \centering
                        Figure 3
                  \end{figure}
                  A model of giving recommendations for decision support. 
                  
                  This assumes that decision makers will carefully consider recommendations. However, empirical evidence suggests this is not the case.
            \end{column}%
            \hfill%
            \begin{column}{.25\textwidth}
                  \scriptsize
                  \begin{itemize}
                        \item[\xmark] Options
                        \item[\xmark] Possibilities
                        \item[\xmark] Judgement \& Value
                        \item[\xmark] Trade-offs
                        \item[\xmark] Understandable
                  \end{itemize}
            \end{column}%
      \end{columns}
\end{frame}

\begin{frame}
      \frametitle{Giving recommendations with cognitive forcing}
      \begin{columns}[T] % align columns
            \begin{column}{.75\textwidth}
                  \begin{figure}[htbp]
                        \centering
                        Figure 3
                  \end{figure}
                  A model of giving recommendations for decision support. 
                  
                  This assumes that decision makers will carefully consider recommendations. However, empirical evidence suggests this is not the case.
            \end{column}%
            \hfill%
            \begin{column}{.25\textwidth}
                  \scriptsize
                  \begin{itemize}
                        \item[\xmark] Options
                        \item[\xmark] Possibilities
                        \item[\xmark] Judgement \& Value
                        \item[\xmark] Trade-offs
                        \item[\xmark] Understandable
                  \end{itemize}
            \end{column}%
      \end{columns}
\end{frame}



\End
\begin{frame}
      \frametitle{Bibliography}
      Template: \href{https://github.com/piazzai/arguelles}{Here \faGithub} \\
\end{frame}


\section{Template}

\begin{frame}[plain]
      \frametitle{A plain frame has no headline}
      \begin{table}
            \small
            \begin{tabular}{rl}
                  \ttfamily\textbackslash Alegreya              & \Alegreya Lorem ipsum dolor sit amet              \\
                  \ttfamily\textbackslash AlegreyaMedium        & \AlegreyaMedium Lorem ipsum dolor sit amet        \\
                  \ttfamily\textbackslash AlegreyaExtraBold     & \AlegreyaExtraBold Lorem ipsum dolor sit amet     \\
                  \ttfamily\textbackslash AlegreyaBlack         & \AlegreyaBlack Lorem ipsum dolor sit amet         \\
                  \ttfamily\textbackslash AlegreyaSansThin      & \AlegreyaSansThin Lorem ipsum dolor sit amet      \\
                  \ttfamily\textbackslash AlegreyaSansLight     & \AlegreyaSansLight Lorem ipsum dolor sit amet     \\
                  \ttfamily\textbackslash AlegreyaSans          & \AlegreyaSans Lorem ipsum dolor sit amet          \\
                  \ttfamily\textbackslash AlegreyaSansMedium    & \AlegreyaSansMedium Lorem ipsum dolor sit amet    \\
                  \ttfamily\textbackslash AlegreyaSansExtraBold & \AlegreyaSansExtraBold Lorem ipsum dolor sit amet \\
                  \ttfamily\textbackslash AlegreyaSansBlack     & \AlegreyaSansBlack Lorem ipsum dolor sit amet
            \end{tabular}
      \end{table}
      \vfill
      \begin{alert}{Alert!}
            A \textit{plain} frame does not show the progress bar but still appears in it unless the frame comes after \texttt{\textbackslash End}
      \end{alert}
\end{frame}

\begin{frame}[standout]
      \centering\large
      A \textbf{\itshape\scshape standout} frame can be used to focus attention
\end{frame}

\End
\begin{frame}[plain,standout]
      \centering
      In combination with \textit{plain}, \\
      it makes a nice thank-you slide!
      \vfill
      \scalebox{4}{\faGithub} \par\bigskip
      \url{https://github.com/piazzai/arguelles} \\
      \url{https://ctan.org/pkg/beamertheme-arguelles}
\end{frame}

\end{document}