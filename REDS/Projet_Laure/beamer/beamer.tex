%----------------------------------------------------------------------------------------
%	PACKAGES AND THEMES
%----------------------------------------------------------------------------------------
\documentclass[aspectratio=169,xcolor=dvipsnames]{beamer}
\usetheme{SimplePlus}

\usepackage{hyperref}
\usepackage{graphicx} % Allows including images
\usepackage{booktabs} % Allows the use of \toprule, \midrule and \bottomrule in tables
\graphicspath{{figs/}}
\usepackage{tikz}
\newcommand{\xmark}{%
\tikz[scale=0.15] {
      \draw[line width=0.7,line cap=round] (0,0) to [bend left=6] (1,1);
      \draw[line width=0.7,line cap=round] (0.2,0.95) to [bend right=3] (0.8,0.05);
}}
\newcommand{\cmark}{%
\tikz[scale=0.15] {
      \draw[line width=0.7,line cap=round] (0.25,0) to [bend left=10] (1,1);
      \draw[line width=0.8,line cap=round] (0,0.35) to [bend right=1] (0.23,0);
}}




%----------------------------------------------------------------------------------------
%	TITLE PAGE
%----------------------------------------------------------------------------------------

\title[NAS]{Neural Architecture Search} % The short title appears at the bottom of every slide, the full title is only on the title page
\subtitle{REDS}

\author[Charles, Mathis] {Charkes VIN, Mathis KOROGLU}

\institute[SU] % Your institution as it will appear on the bottom of every slide, may be shorthand to save space
{
    Sorbonne Université % Your institution for the title page
}
\date{\today} % Date, can be changed to a custom date


%----------------------------------------------------------------------------------------
%	PRESENTATION SLIDES
%----------------------------------------------------------------------------------------

\begin{document}

\begin{frame}
    % Print the title page as the first slide
    \titlepage
\end{frame}

\begin{frame}{Overview}
    % Throughout your presentation, if you choose to use \section{} and \subsection{} commands, these will automatically be printed on this slide as an overview of your presentation
    \tableofcontents
\end{frame}

%------------------------------------------------
\section{Neural architecture search}
%------------------------------------------------

\begin{frame}{}
    \begin{figure}[htbp]
        \centering
        \includegraphics[width=\linewidth]{overview_NAS.pdf}
        \caption{Overview of NAS.

            A search strategy iteratively selects architectures (typically by using an architecture encoding method) from a predened search space $ \mathcal{A} $ .

            The architectures are passed to a performance estimation strategy, which returns the performance estimate to the search strategy.

            For one-shot methods, the search strategy and performance estimation strategy are inherently coupled.}
        % citer le papier 1000
    \end{figure}
\end{frame}

%------------------------------------------------

\begin{frame}{Search space}.
    \begin{block}{Definition}
        The set of all archtectures that the NAS algorithm is allowed to select.
    \end{block}
    \begin{itemize}
        \item Size: from a few thousand to over $ 10^20 $.
        \item Reduction: adding domain knowledge.
        \item [$\rightarrow$] Introduce humain bias $\rightarrow \xmark $ reduce the chance of finding truly nover architecture.
    \end{itemize}
\end{frame}

\begin{frame}{Search strategy}
    \begin{block}{Definition}
        The optimization technique used to find a high-performing architecture in the search space.
    \end{block}
    \begin{itemize}
        \item Black-box optimization techniques : RL, Bayesian optimization, evolutionary search.
        \item One-shot techniques: supernet-hypernet based methods.
    \end{itemize}
    % Naturellement, certaine technique ne rentre dans aucunes des cases 
\end{frame}

\begin{frame}{Performance estimation strategy}
    \begin{block}{Definition}
        Any method used to quickly predict the performance of neural architectures in order to avoid fully training the architecture.
    \end{block}
    \begin{itemize}
        \item Full training \& evaluation.
        \item Performance estimation strategy.
    \end{itemize}
    \begin{figure}[htbp]
        \centering
        \includegraphics[width=.7\linewidth]{perf_estim_strat.pdf}
        % citer le papier 1000
    \end{figure}
    % parler du meta learning (dernier paragraphe page 26), j'pense c'est proche de difusionNAG 
\end{frame}

%------------------------------------------------

\begin{frame}{Multiple Columns}
    \begin{columns}[c] % The "c" option specifies centered vertical alignment while the "t" option is used for top vertical alignment

        \column{.45\textwidth} % Left column and width
        \textbf{Heading}
        \begin{enumerate}
            \item Statement
            \item Explanation
            \item Example
        \end{enumerate}

        \column{.5\textwidth} % Right column and width
        Lorem ipsum dolor sit amet, consectetur adipiscing elit. Integer lectus nisl, ultricies in feugiat rutrum, porttitor sit amet augue. Aliquam ut tortor mauris. Sed volutpat ante purus, quis accumsan dolor.

    \end{columns}
\end{frame}

%------------------------------------------------
\section{Our contribution}
%------------------------------------------------

\begin{frame}{}

\end{frame}

%------------------------------------------------

\begin{frame}{}

\end{frame}

%------------------------------------------------

\begin{frame}{}

\end{frame}

%------------------------------------------------

\begin{frame}{References}
    % Beamer does not support BibTeX so references must be inserted manually as below
    \footnotesize{
        \begin{thebibliography}{99}
            \bibitem[Smith, 2012]{p1} John Smith (2012)
            \newblock Title of the publication
            \newblock \emph{Journal Name} 12(3), 45 -- 678.
        \end{thebibliography}
    }
\end{frame}

%------------------------------------------------

\begin{frame}
    \Huge{\centerline{\textbf{Thank you}}}
\end{frame}

%----------------------------------------------------------------------------------------

\end{document}

% \begin{frame}[fragile] % Need to use the fragile option when verbatim is used in the slide
%     \frametitle{frame documnetation}
%     An example of the \verb|\cite| command to cite within the presentation:\\~

%     This statement requires citation \cite{p1}.

%     In this slide, some important text will be \alert{highlighted} because it's important. Please, don't abuse it.

%     \begin{block}{Block}
%         Sample text
%     \end{block}

%     \begin{alertblock}{Alertblock}
%         Sample text in red box
%     \end{alertblock}

%     \begin{examples}
%         Sample text in green box. The title of the block is ``Examples".
%     \end{examples}
% \end{frame}